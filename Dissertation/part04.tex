\chapter{Исследования в области формирования единичной микрованны расплава и основного материала применительно к методу селективного лазерного сплавления}\label{ch:chapter4}

\section{Математическое моделирование температурного поля, возникающего в материале, при различных энерго-скоростных параметрах воздействия лазерного излучения}\label{sec:chapter4/section1}

Моделирование температурного поля. 

\section{Экспериментальные исследования в области формирования единичной микрованны расплава при послойном сплавлении металлопорошковой композиции}\label{sec:chapter4/section2}

Определение размера "трека" полученного при различной мощности и скорости сканирования лазера.

\section{Адаптация математической модели с целью прогнозирования величины микрованны расплава в зависимости от энерго-скоростных параметров воздействия лазерного излучения}\label{sec:chapter4/section3}

Расчёт коэффициента поглощающей способности. Верификация математической модели с экспериментальными данными.

\section{Экспериментальные исследования в области формирования основного материала при послойном сплавлении металлопорошковой композиции}\label{sec:chapter4/section4}

Трехфакторный эксперимент. Трехмерная модель зависимости пористости синтезированного материала от мощности, скорости сканирования и межтрекового расстояния.
