\chapter{Исследование структурно-фазового состояния и физико-механических свойств литого и синтезированного материала после воздействия различных видов и режимов термической обработки}\label{ch:chapter6}

\section{Исследование фазовых превращений в литом и синтезированном материале методом дифференциальной сканирующей калориметрии}\label{sec:chapter6/section1}

ДСК литого и синтезированного материала при различных скоростях нагрева. Сравнительный анализ результатов.

\section{Исследование структурно-фазового состояния и физико-механических свойств литого и синтезированного материала после воздействия различных режимов отжига}\label{sec:chapter6/section2}

СЭМ, ПЭМ, РСМА синтезированного материала после отжига. Сравнение с синтезированным материалом AlSi10Mg и АК9ч.

\section{Исследование структурно-фазового состояния и физико-механических свойств литого и синтезированного материала после закалки и естественного старения}\label{sec:chapter6/section3}

СЭМ, ПЭМ, РСМА синтезированного материала после закалки и естественного старения. Сравнение с синтезированным материалом AlSi10Mg и АК9ч. Критическая скорость охлаждения, распад твёрдого раствора.

\section{Исследование структурно-фазового состояния и физико-механических свойств литого и синтезированного материала после закалки и искусственного старения}\label{sec:chapter6/section3}

СЭМ, ПЭМ, РСМА синтезированного материала после закалки и искусственного старения. Кинетика старения. Сравнение с синтезированным материалом AlSi10Mg и АК9ч.